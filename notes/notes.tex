\documentclass[a4paper,12pt,twoside]{report}

\usepackage[english]{babel}
\usepackage[utf8]{inputenc}

%------------------------------
% Header settings for all pages
%------------------------------
\usepackage{fancyhdr}

\pagestyle{fancy}
\fancyhf{}
\rhead{Kian Nejadfard}
\lhead{Research Kernel}
%------------------------------

%------------------------------
% For timeline
%------------------------------
\newcommand{\foo}{\hspace{-2.3pt}$\bullet$ \hspace{5pt}}
\usepackage{graphicx}
%------------------------------

%------------------------------
% Bibliography configurations
%------------------------------
\usepackage[backend=biber, sorting=none]{biblatex}
\bibliography{notes.bib}
%------------------------------

%------------------------------
% Customizations for lstlisting package
%------------------------------
\usepackage{listings}
\usepackage[utf8]{inputenc}

\usepackage{xcolor}

\definecolor{codegreen}{rgb}{0,0.6,0}
\definecolor{codegray}{rgb}{0.5,0.5,0.5}
\definecolor{codepurple}{rgb}{0.58,0,0.82}
\definecolor{backcolour}{rgb}{0.95,0.95,0.92}

\lstdefinestyle{mystyle}{
	backgroundcolor=\color{backcolour},
	commentstyle=\color{codegreen},
	keywordstyle=\color{magenta},
	numberstyle=\tiny\color{codegray},
	stringstyle=\color{codepurple},
	basicstyle=\ttfamily\footnotesize,
	breakatwhitespace=false,
	breaklines=true,
	captionpos=b,
	keepspaces=true,
	numbers=left,
	numbersep=5pt,
	showspaces=false,
	showstringspaces=false,
	showtabs=false,
	tabsize=2
}

\lstset{style=mystyle}
%------------------------------

\usepackage{fancyvrb}
%------------------------------

%------------------------------
% Table Configurations
%------------------------------
\usepackage{multirow}
% Spacing between cell content and left/right borders
\setlength{\tabcolsep}{5pt}
% The height of each row is set to 1.5 relative to its default height.
\renewcommand{\arraystretch}{1.3}

% This is to be able to have tables stay within the section they are written at.
\usepackage{float}

\usepackage{longtable}
%------------------------------

\title{Research Kernel}
\author{Kian Nejadfard}

\begin{document}
    \maketitle

    \begin{abstract}
    	RKern is my new research project for developing a kernel from scratch in C++ to evaluate how the use of zero-overhead abstraction mechanisms can affect a kernel project in terms of codebase maintainability, modularity, usability, and performance.
    \end{abstract}

	\chapter{Preface}

	    \section{Motivation}
	    	I have always been interested in how software and hardware work together. Naturally this translated into me being curious about kernels. After all, a kernel is the entity that controls and provides mechanisms for accessing resources by user-space software.

	    	After going through my master's degree coursework, I became more familiar with topics such as the Linux kernel's programming interface along with how parts of the kernel work internally, the assembly language (I mostly used \textit{NASM}), computer and instruction set architecture, how CPUs work internally, basics of compilers and language design.

	    	All of a sudden, I noticed something: the board is set for me to follow two of my lifelong interests. \textit{Developing a kernel from scratch}, and \textit{working with embedded systems}.
	    	It was at this point that I felt more confident about all of this. I could finally understand what's going on at the hardware level, and what it would take for software to "talk" to hardware.

	    \section{Goals}
	        My primary goals for this project and paper are:
	        \begin{itemize}
	            \item \textbf{Learn more about the internals of kernels}.
	            I am a firm believer in learning by doing. Therefore, in my opinion, the best method for learning how kernels work is by making one from scratch. Of course, this is re-inventing a wheel that has been re-invented by many other people so far. However, I see a lot of value in doing this.

	            \item \textbf{Putting a claim to test: Low-overhead abstraction mechanisms in the kernel}.
	            Given that I have always been interested in C++ and have worked with this language a lot since the day that I started programming for the first time. I have been searching for reasons why most kernels are developed in C, and not C++. I have a hypothesis that using C++ and its low-overhead abstraction mechanisms can add many values to kernel development. Times have changed since the 1990s, and C++ has evolved a lot. Along with C++, the tooling has also evolved a lot. When searching for an answer to \textit{"Why basically all kernels in use today are developed with C and not C++?"}, I have come across a lot of rants about how unfit C++ is for such a task, how unreliable the C++ compilers are, and similar negative talks. I am putting such arguments to test in this research.
	            % REFINE THE PREVIOUS PARAGRAPH AND ADD REFERENCES? MAYBE REPHRASE A BIT.
	        \end{itemize}

	    \section*{Timeline}
	        The following is a timeline of key events in this project:
	        \begin{itemize}
	            \item \textbf{August 2018} - The idea started.
	            \item \textbf{September/October 2018} - Built a home office and a PC.
	            \item \textbf{January 2019} - Started research on kernels and operating systems, and created the source repository. I used \textit{osdev.org}\cite{osdev} extensively to learn about the steps it takes to write a basic kernel from scratch.
	            \item \textbf{January 2020} - Finished master's degree courses, no thesis work done yet. The kernel research project has been quiet for a year now as I could not find time to make progress.
	            \item \textbf{February 2020} - Changed jobs,
	            \item \textbf{November 2020} - Started working on the kernel research project again. Chose the name \textit{RKern} for it, and set the project vision.
	        \end{itemize}

    \chapter{Kernels, What They Are, And How They Differ}

	    \section{What Is a Kernel?}
	    	TODO

	    \section{Kernel Types}
	    	TODO

	    \section{A Brief History}
	        TODO: review how Unix, Linux, BSDs, etc. were started. Including a timeline, and also review the timeline of Assembly, C, and C++. The idea is to find out if the main reason for today's kernels' use of C is just the time that they started development and the fact that C was the best and only sane choice at the time. Clearly, after decades of development, you can't just change languages of the kernel. So... maybe it is time for a fresh start!

    \chapter{Prerequisites And Development Setup}

	    \section{Cross-Compiling}
	        Since I am compiling the RKern source code on an x86\_64 machine (host) while targeting different architectures (e.g. x86\_32 or RV) (correct wording?), I need to use a cross-compiler.

	        \subsection{LLVM}
	        If using the LLVM toolchain, it is much easier to cross-compile to different targets mainly due to the fact that you don't need to setup anything differently. Just the fact that you have LLVM set up means you can use compilation targets other than your host machine.
	        For example, to use clang++ to compile a C++ source file for i386 architecture, the \verb|--target=i686-pc-none-elf| flag can be used. Similarly, to target the 32-bit RISC-V architecture, \verb|--target=riscv32-unknown-elf| can be used.

			\subsection{GCC}
	        TODO: add notes for setting up cross-compiler for GCC toolchain.

	    \section{Building .iso Images}
	    TODO
	    Dependencies: grub-mkrescue, xorriso

    \chapter{Binary File Analysis}

        \section{Disassembling The Binary}

            \begin{verbatim}
llvm-objdump --disassemble-all rkern.bin
objdump -d rkern.bin
            \end{verbatim}

    \chapter{Linkers and Linker Scripts}

        \section{What Do Linkers Do?}
            TODO: use reference http://www.bravegnu.org/gnu-eprog/linker.html

        \section{Linker Script}
            A linker script can define 4 pieces of information:
            \begin{enumerate}
            	\item \textbf{Memory layout}
            	\item \textbf{Section definitions} - Defines the structure of the binary file that will be produced by the linker program.
            	\item \textbf{Options} - Specifications of architecture, entry point, etc. if needed.
            	\item \textbf{Symbols} - Variables that have to be injected into the program at link time.
            \end{enumerate}\cite{memfaultLinkerScripts}

            \subsection{Memory Layout}
                In order to allocate program space, the linker needs to know how much memory is available, and at what addresses that memory exists. This is what the \verb|MEMORY| definition in the linker script is for.

                The syntax for \verb|MEMORY| is as follows:
                \begin{verbatim}
MEMORY
{
    name [(attr)] : ORIGIN = origin, LENGTH = len
    ...
}
                \end{verbatim}
            	Where:
            	\begin{itemize}
            		\item \verb|name| is the region's name. The choice of name is arbitrary as they do not carry any specific meaning. Typical names include \textbf{flash} and \textbf{ram}.
            		\item \verb|(attr)| are optional attributes for the region, such as \verb|w| (writable), \verb|r| (readable), \verb|x| (executable). Flash memory is usually \verb|rx| while ram is usually \verb|wrx|. Notice that these attributes do not actually set memory, rather they just describe the properties of the memory region.
            		\item \verb|origin| is the start address of the memory region.
            		\item \verb|len| is the size of the memory region in bytes.
            	\end{itemize}

            \subsection{Program Headers}
                Also known as \textit{segments}, the \textit{program headers} describe how a program is loaded in memory from an ELF object file format. While the linker creates reasonable program headers by default, sometimes it may be necessary to customize them.\cite{gnuldProgramHeaders}

                In order to observe the program headers of an ELF file, the following command may be used:
\begin{verbatim}
objdump -p <elf file>
\end{verbatim}

                Program headers may be defined by using the \lstinline|PHDRS| command in the linker script:
\begin{verbatim}
PHDRS
{
    name type [ FILEHDR ] [ PHDRS ] [ AT ( address ) ]
[ FLAGS ( flags ) ] ;
}
\end{verbatim}

                Certain program header types describe segments of memory which are loaded from the ELF file by the system loader. In the linker script, the contents of these segments are specified by directing allocated output sections to be placed in the segment. To do this, the command describing the output section in the \lstinline|SECTIONS| command should use \lstinline|:name|, where \lstinline|name| is the program header name as it appears in the \lstinline|PHDRS| command.\cite{gnuldProgramHeaders}

                If a section is placed in one or more segments using `:name', then all subsequent allocated sections which do not specify `:name' are placed in the same segments.\cite{gnuldProgramHeaders}

                The \lstinline|FILEHDR| and \lstinline|PHDRS| keywords which may appear after the program header type also indicate contents of the segment of memory. The \lstinline|FILEHDR| keyword means that the segment should include the ELF file header. The \lstinline|PHDRS| keyword means that the segment should include the ELF program headers themselves.\cite{gnuldProgramHeaders}

                \lstinline|type| may be one of the following\cite{gnuldProgramHeaders}:
                \begin{itemize}
                    \item \lstinline|PT_NULL| - Indicates an unused program header.
                    \item \lstinline|PT_LOAD| - Indicates that this program header describes a segment to be loaded from the file.
                    \item \lstinline|PT_DYNAMIC| - Indicates a segment where dynamic linking information can be found.
                    \item \lstinline|PT_INTERP| - Indicates a segment where the name of the program interpreter may be found.
                    \item \lstinline|PT_NOTE| - Indicates a segment holding note information.
                    \item \lstinline|PT_SHLIB| - A reserved program header type, defined but not specified by the ELF ABI.
                    \item \lstinline|PT_PHDR| - Indicates a segment where the program headers may be found.
                    \item \lstinline|expression| - An expression giving the numeric type of the program header. This may be used for types not defined above.
                \end{itemize}

                It is possible to specify that a segment should be loaded at a particular address in memory. This is done using an \lstinline|AT| expression. This is identical to the \lstinline|AT| command used in the \lstinline|SECTIONS| command. Using the \lstinline|AT| command for a program header overrides any information in the \lstinline|SECTIONS| command.\cite{gnuldProgramHeaders}

                Knowing the above, the following is a simple program header definition for use with HiFive Rev B development board:
\begin{lstlisting}
PHDRS
{
    flash PT_LOAD;
    ram PT_NULL;
}
\end{lstlisting}

            \subsection{Sections}

            \subsection{Options}

            \subsection{Symbols}

        \chapter{HiFive1 Rev B}
            After learning about what linkers do and what a linker script is composed of, we should consult the documentation of the hardware that we want to work with, in order to figure out critical information that we need to use for the compilation and linking process.

            \section{Gathering Hardware Information}
            After reviewing the documentations of the \textit{HiFive1 Rev B} board, as well as the \textit{FE310-G002} core that comes with it, I have discovered the following information:
            \begin{itemize}
            	\item The FE310-G002 core is configured to support the RV32IMAC ISA options. This specifies the architecture to be used when producing the binary file.\cite{hifive1RevBConfig}

				\item The data SRAM is 16 KiB.\cite{hifive1RevBConfig}

            	\item The system mask ROM is 8 KiB in size and contains simple boot code.\cite{hifive1RevBConfig}

                \item The mask ROM (MROM) is fixed at design time, and is located on the peripheral bus on FE310-G002, but instructions fetched from MROM are cached by the core’s I-cache. The MROM contains an instruction at address 0x10000 which jumps to the OTP start address at 0x20000.\cite{fe310g002manBootProcess}

                \item A dedicated quad-SPI (QSPI) flash interface is provided to hold code and data for the system. The QSPI interface supports burst reads of 32 bytes over TileLink to accelerate instruction cache refills. The QSPI can be programmed to support eXecute-In-Place modes to reduce SPI command overhead on instruction cache refills. The QSPI interface also supports single-word data reads over the primary TileLink interface, as well as programming operations using memory-mapped control registers.\cite{hifive1RevBConfig}

            	\item FE310-G002 boots by jumping to the beginning of the OTP memory and executing the code found there. As shipped, the OTP memory at the boot location is programmed to jump immediately to the end of the OTP memory, which in turn jumps to the beginning of the SPI Flash at \textbf{0x20000000}.\cite{hifive1RevBBootCode}

                \item The OTP is located on the peripheral bus, with both a control register interface to program the OTP, and a memory read port interface to fetch words from the OTP. Instruction fetches from the OTP memory read port are cached in the E31 core’s instruction cache.\cite{fe310g002manBootProcess}

                \item The HiFive1 Rev B Board is shipped with a modifiable boot loader at the beginning of SPI Flash (0x20000000). At the end of this program’s execution the core jumps to the main user portion of code at \textbf{0x20010000}. This program is designed to allow quick boot, but also a safe reboot option if a “bad” program is flashed into the SPI Flash. A bad program is one which makes it impossible for the programmer to communicate with the board. For example, a program which disables FE310’s active clock, or which puts the core to sleep with no way of waking it up. Bad programs can always be restarted using the RESET button, and using the “safe” bootloader can be halted before they perform any unsafe behavior.\cite{hifive1RevBBootLoader}

            	\item To activate normal boot mode, press the RESET button on the HiFive1 Rev B. After approximately 1 second, the green LED will flash for 1/2 second, then the user program will execute.\cite{hifive1RevBBootLoader}

                \item To activate safe boot mode, press the RESET button. When the green LED flashes, immediately press the RESET button again. After 1 second, the red LED will blink. The user program will not execute, and the programmer can connect. To exit “safe” boot mode, press the RESET button a final time.\cite{hifive1RevBBootLoader}

                \item There are 3 serial peripheral interface (SPI) controllers. Each controller provides a means for serial communication between the FE310-G002 and off-chip devices, like quad-SPI Flash memory. Each controller supports master-only operation over single-lane, dual-lane, and quad-lane protocols. Each controller supports burst reads of 32 bytes over TileLink to accelerate instruction cache refills. 1 SPI controller can be programmed to support eXecute-In-Place (XIP) modes to reduce SPI command overhead on instruction cache refills.\cite{fe310g002manOverview}

                \item Two universal asynchronous receiver/transmitter (UARTs) are available and provide a means for serial communication between the FE310-G002 and off-chip devices.\cite{fe310g002manOverview}

                \item The FE310-G002 has an I2C controller to communicate with external I2C devices, such as sensors, ADCs, etc.\cite{fe310g002manOverview}

                \item The FE310-G002 provides external debugger support over an industry-standard JTAG port, including 8 hardware-programmable breakpoints per hart.\cite{fe310g002manOverview}
            \end{itemize}

            \section{Memory Map}
            	The following tables show the FE310-G002 memory map along with relevant attributes:\cite{fe310g002manMMap}

            	\begin{table}[H]
            		\centering
            		\begin{tabular}{| p{3cm} | c | c | p{3cm} |}
            			\hline
            			\textbf{Description} & \textbf{Base} & \textbf{Top} & \textbf{Attributes}\\
            			\hline
            			\hline
            			Debug & 0x0000\_0000 & 0x0000\_0FFF & RWX A\\
            			\hline
            		\end{tabular}
            		\caption{Debug Address Space}
            	\end{table}

            	\begin{table}[H]
            		\centering
            		\begin{tabular}{| p{4cm} | c | c | p{3cm} |}
            			\hline
            			\textbf{Description} & \textbf{Base} & \textbf{Top} & \textbf{Attributes}\\
            			\hline
            			Mode Select & 0x0000\_1000 & 0x0000\_1FFF & R XC\\
            			Reserved & 0x0000\_2000 & 0x0000\_2FFF & \\
            			Error Device & 0x0000\_3000 & 0x0000\_3FFF & RWX A\\
            			Reserved & 0x0000\_4000 & 0x0000\_FFFF & \\
            			Mask ROM & 0x0001\_0000 & 0x0001\_1FFF & R XC\\
            			Reserved & 0x0001\_2000 & 0x0001\_FFFF & \\
            			OTP Memory Region & 0x0002\_0000 & 0x0002\_1FFF & R XC\\
            			Reserved & 0x0002\_2000 & 0x001F\_FFFF & \\
            			\hline
            		\end{tabular}
            		\caption{On-Chip Non Volatile Memory}
            	\end{table}

                \begin{longtable}[H]{| p{4.5cm} | p{3cm} | p{3cm} | p{2.5cm} |}
               		\hline
               		\textbf{Description} & \textbf{Base} & \textbf{Top} & \textbf{Attributes}\\
               		\hline
               		\hline
               		\endfirsthead
                    CLINT & 0x0200\_0000 & 0x0200\_FFFF & RW A\\
                    Reserved & 0x0201\_0000 & 0x07FF\_FFFF & \\
                    E31 ITIM (8 KiB) & 0x0800\_0000 & 0x0800\_1FFF & RWX A\\
                    Reserved & 0x0800\_2000 & 0x0BFF\_FFFF & \\
                    PLIC & 0x0C00\_0000 & 0x0FFF\_FFFF & RW A\\
                    AON & 0x1000\_0000 & 0x1000\_0FFF & RW A\\
               		Reserved & 0x1000\_1000 & 0x1000\_7FFF & \\
               		PRCI & 0x1000\_8000 & 0x1000\_8FFF & RW A\\
               		Reserved & 0x1000\_9000 & 0x1000\_FFFF & \\
               		OTP Control & 0x1001\_0000 & 0x1001\_0FFF & RW A\\
               		Reserved & 0x1001\_1000 & 0x1001\_1FFF & \\
               		GPIO & 0x1001\_2000 & 0x1001\_2FFF & RW A\\
               		UART 0 & 0x1001\_3000 & 0x1001\_3FFF & RW A\\
               		QSPI 0 & 0x1001\_4000 & 0x1001\_4FFF & RW A\\
               		PWM 0 & 0x1001\_5000 & 0x1001\_5FFF & RW A\\
               		I2C 0 & 0x1001\_6000 & 0x1001\_6FFF & RW A\\
               		Reserved & 0x1001\_7000 & 0x1002\_2FFF & \\
               		UART 1 & 0x1002\_3000 & 0x1002\_3FFF & RW A\\
               		SPI 1 & 0x1002\_4000 & 0x1002\_4FFF & RW A\\
               		PWM 1 & 0x1002\_5000 & 0x1002\_5FFF & RW A\\
               		Reserved & 0x1002\_6000 & 0x1003\_3FFF & \\
               		SPI 2 & 0x1003\_4000 & 0x1003\_4FFF & RW A\\
               		PWM 2 & 0x1003\_5000 & 0x1003\_5FFF & RW A\\
               		Reserved & 0x1003\_6000 & 0x1FFF\_FFFF & \\
               		\hline
               		\caption{On-Chip Peripherals}
                \end{longtable}

            	\begin{table}[H]
            		\centering
            		\begin{tabular}{| p{4.5cm} | p{3cm} | p{3cm} | p{2.5cm} |}
            			\hline
            			\textbf{Description} & \textbf{Base} & \textbf{Top} & \textbf{Attributes}\\
            			\hline
            			\hline
            			QSPI 0 Flash (512 MiB) & 0x2000\_0000 & 0x3FFF\_FFFF & R XC\\
            			Reserved & 0x4000\_0000 & 0x7FFF\_FFFF & \\
            			\hline
            		\end{tabular}
            		\caption{Off-Chip Non-volatile Memory}
            	\end{table}

            	\begin{table}[H]
            		\centering
            		\begin{tabular}{| p{4.5cm} | p{3cm} | p{3cm} | p{2.5cm} |}
            			\hline
            			\textbf{Description} & \textbf{Base} & \textbf{Top} & \textbf{Attributes}\\
            			\hline
            			\hline
            			E31 DTIM (16 KiB) & 0x8000\_0000 & 0x8000\_3FFF & RWX A\\
            			Reserved & 0x8000\_4000 & 0xFFFF\_FFFF & \\
            			\hline
            		\end{tabular}
            		\caption{On-Chip Volatile Memory}
            	\end{table}

    \printbibliography
\end{document}
