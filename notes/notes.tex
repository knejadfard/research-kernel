\documentclass[a4paper,12pt,twoside]{book}

\usepackage[english]{babel}
\usepackage[utf8]{inputenc}

%------------------------------
% Header settings for all pages
%------------------------------
\usepackage{fancyhdr}

\pagestyle{fancy}
\fancyhf{}
\rhead{Kian Nejadfard}
\lhead{Research Kernel}
%------------------------------

%------------------------------
% For timeline
%------------------------------
\newcommand{\foo}{\hspace{-2.3pt}$\bullet$ \hspace{5pt}}
\usepackage{graphicx}
%------------------------------

% --------------------
% Bibliography configurations
% --------------------
\usepackage[backend=biber, sorting=none]{biblatex}
\bibliography{notes.bib}

\title{Research Kernel}
\author{Kian Nejadfard}

\begin{document}
    \maketitle
    
    \section*{Motivation}
        My main motivations for starting this project are:
        \begin{itemize}
            \item \textbf{Learn more about the internals of kernels}.
            
            I am a firm believer in learning by doing. Therefore, in my opinion, the best method for learning how kernels work is by making one from scratch. Of course, this is re-inventing a wheel that has been re-invented by many other people so far. However, I see a lot of value in doing so.
            \item \textbf{Putting an idea to test: C++ for kernel development}.
            
            Given that I have always been interested in C++ and have worked with this language a lot since the day that I started programming for the first time, I have been searching for reasons why most kernels (well, perhaps all mainstream kernels today?) are developed in C, and not C++. I have a hypothesis that, using C++ can bring many values for kernel development. Times have changed since the 1990s, and C++ has evolved a lot. Along with C++, the tooling has also evolved a lot. When searching for an answer to "Why basically all kernels in use today are developed with C and not C++?", I have come across a lot of rants about how unfit C++ is for such a task, how unreliable the C++ compilers are, and similar negative talks.
            % REFINE THE PREVIOUS PARAGRAPH AND LINK TO CHAPTER ONE, WHICH HAS REFERENCES TO VARIOUS DISCUSSIONS, LITERATURE, ETC.
            
            I believe that the C++ community has come a very long way since the early days of this language, and that it is time to put this idea to test. I believe that C++ has quite a bit to offer in terms of expressiveness and language constructs, that can benefit kernel development. 
        \end{itemize}

    \section*{Timeline}
        The following is a timeline of key events in this project:
        \begin{itemize}
            \item \textbf{August 2018} - The idea started.
            \item \textbf{September/October 2018} - Built a home office and a PC.
            \item \textbf{January 2019} - Started research on kernels and operating systems, and created the source repository. I used \textit{osdev.org}\cite{osdev} extensively to learn about the steps it takes to write a basic kernel from scratch.
            \item \textbf{January 2020} - Finished master's degree courses, no thesis work done yet. The kernel research project has been quiet for a year now as I could not find time to make progress.
            \item \textbf{February 2020} - Changed jobs,
            \item \textbf{November 2020} - Started working on the kernel research project again. Chose the name \textit{RKern} for it, and set the project vision.
        \end{itemize}
    
    \chapter{History of Today's Kernels}
        TODO: review how Unix, Linux, BSDs, etc. were started. Including a timeline, and also review the timeline of Assembly, C, and C++. The idea is to find out if the main reason for today's kernels' use of C is just the time that they started development and the fact that C was the best and only sane choice at the time. Clearly, after decades of development, you can't just change languages of the kernel. So... maybe it is time for a fresh start!

    \chapter{The LLVM Toolchain}
        This chapter talks about the LLVM toolchain and how it can be used for cross-compiling for different target architectures, which is essential when developing kernels.
    
    \section{Cross-compiling with LLVM/Clang}
    
    \printbibliography
\end{document}