\section{Motivation}
I have always been interested in how software and hardware work together. Naturally this translated into me being curious about kernels. After all, a kernel is the entity that controls and provides mechanisms for accessing resources by user-space software.

After going through my master's degree coursework, I became more familiar with topics such as the Linux kernel's programming interface along with how parts of the kernel work internally, the assembly language (I mostly used \textit{NASM}), computer and instruction set architecture, how CPUs work internally, basics of compilers and language design.

All of a sudden, I noticed something: the board is set for me to follow two of my lifelong interests. \textit{Developing a kernel from scratch}, and \textit{working with embedded systems}.
It was at this point that I felt more confident about all of this. I could finally understand what's going on at the hardware level, and what it would take for software to "talk" to hardware.

\section{Goals}
My primary goals for this project and paper are:
\begin{itemize}
    \item \textbf{Learn more about the internals of kernels}.
    I am a firm believer in learning by doing. Therefore, in my opinion, the best method for learning how kernels work is by making one from scratch. Of course, this is re-inventing a wheel that has been re-invented by many other people so far. However, I see a lot of value in doing this.

    \item \textbf{Putting a claim to test: Low-overhead abstraction mechanisms in the kernel}.
    Given that I have always been interested in C++ and have worked with this language a lot since the day that I started programming for the first time. I have been searching for reasons why most kernels are developed in C, and not C++. I have a hypothesis that using C++ and its low-overhead abstraction mechanisms can add many values to kernel development. Times have changed since the 1990s, and C++ has evolved a lot. Along with C++, the tooling has also evolved a lot. When searching for an answer to \textit{"Why basically all kernels in use today are developed with C and not C++?"}, I have come across a lot of rants about how unfit C++ is for such a task, how unreliable the C++ compilers are, and similar negative talks. I am putting such arguments to test in this research.
    % REFINE THE PREVIOUS PARAGRAPH AND ADD REFERENCES? MAYBE REPHRASE A BIT.
\end{itemize}

\section*{Timeline}
The following is a timeline of key events in this project:
\begin{itemize}
    \item \textbf{August 2018} - The idea started.
    \item \textbf{September/October 2018} - Built a home office and a PC.
    \item \textbf{January 2019} - Started research on kernels and operating systems, and created the source repository. I used \textit{osdev.org}\cite{osdev} extensively to learn about the steps it takes to write a basic kernel from scratch.
    \item \textbf{January 2020} - Finished master's degree courses, no thesis work done yet. The kernel research project has been quiet for a year now as I could not find time to make progress.
    \item \textbf{February 2020} - Changed jobs,
    \item \textbf{November 2020} - Started working on the kernel research project again. Chose the name \textit{RKern} for it, and set the project vision.
\end{itemize}