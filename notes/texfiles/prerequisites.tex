\section{Cross-Compiling}
Since I am compiling the RKern source code on an x86\_64 machine (host) while targeting different architectures (e.g. x86\_32 or RV) (correct wording?), I need to use a cross-compiler.

\subsection{LLVM Toolchain}
If using the LLVM toolchain, it is much easier to cross-compile to different targets mainly due to the fact that you don't need to setup anything differently. Just the fact that you have LLVM set up means you can use compilation targets other than your host machine.
For example, to use clang++ to compile a C++ source file for i386 architecture, the \verb|--target=i686-pc-none-elf| flag can be used. Similarly, to target the 32-bit RISC-V architecture, \verb|--target=riscv32-unknown-elf| can be used.

\subsection{GNU Toolchain}
This section describes how to build a GNU Toolchain that is capable of compiling source code to RISC-V targets.

Steps:
\begin{enumerate}
    \item Clone source repository of riscv-gnu-toolchain located at \lstinline|git@github.com:riscv/riscv-gnu-toolchain.git|. The \lstinline|--recursive| flag should be used since this repository uses submodules. Otherwise, \lstinline|git submodule update --init --recursive| must be executed after cloning.

    \item There are a number of sub-packages needed for compiling this project. The following command is for installing these dependencies on Debian/Ubuntu: \lstinline|apt-get install autoconf automake autotools-dev curl python3 libmpc-dev libmpfr-dev libgmp-dev gawk build-essential bison flex texinfo gperf libtool patchutils bc zlib1g-dev libexpat-dev|

    \item Configure the project with a prefix directory to ensure it is installed into a known, specific directory of your choice (not mixed with directories provided by operating system's packages): \lstinline|./configure --prefix=/home/kian/binaries/gnu-riscv|

    The default architecture is \lstinline|rv64gc|, with "g" standing for MAFD extensions.

    In order to override the default target architecture, use something line \lstinline|./configure --prefix=/home/kian/binaries/gnu-riscv --with-arch=rv32gc --with-abi=ilp32d|.

    The supported architectures are \lstinline|rv32i| or \lstinline|rv64i| plus standard extensions \textit{(a)tomics}, \textit{(m)ultiplication and division}, \textit{(f)loat}, \textit{(d)ouble}, or \textit{(g)eneral} for \textit{MAFD}.

    Supported ABIs are \lstinline|ilp32| (32-bit soft-float), \lstinline|ilp32d| (32-bit hard-float), \lstinline|ilp32f| (32-bit with single-precision in registers and double in memory, niche use only), \lstinline|lp64|, \lstinline|lp64f| \lstinline|lp64d| (same but with 64-bit long and pointers).

    Since we are intending to use this cross-compiler for building the kernel for HiFive1 Rev B, we need to use the target architecture rv32imac with ilp32 ABI.
\end{enumerate}